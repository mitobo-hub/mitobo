Several of \mitobo's operators as well as the framework itself support individual configuration by the user. For example initial
files or directories the operators should work on can be specified by the user. The probably most common way of individual
configuration is to pass specific path or flag settings to \mitobo\ operators by environment settings as outlined in this chapter.

\section{Environment Variables and Properties}
\mitobo operators support three different ways for user specific configuration:
\begin{itemize}
    \item[a)] environment variables
    \item[b)] properties of the Java virtual machine (JVM) specified with\\
     the option \texttt{'-Dproperty=value'} upon invocation of the JVM
    \item[c)] ImageJ preferences as specified in the file $\;\tilde{ }$/.imagej/IJ\_Prefs.txt\footnote{Please note that this is the
    	default configuration file of the (old) ImageJ framework, in ImageJ $2.0$ its name and place in the file system might change. 
    	Refer to the documentation of the ImageJDev project for details.} 
\end{itemize}
This order reflects the priority of the three options, i.e.~environment variables overwrite JVM
properties, and the latter ones overwrite ImageJ preferences. If for a certain operator no configuration values are
provided by any of these three ways, all default settings of the corresponding internal variables are solely 
depending on the internal settings of each individual operator.

In general there is no limitation for an operator to define configuration variables. Usually they should be properly
documented in the Javadoc of the corresponding class. Some variables of general
interest, however, are listed in the next Section \ref{sec:importantVars} as almost all users might be interested in
using them. 

The naming of the environment variables and properties is not subject to strict rules, i.e.~there are no restrictions in  
\alida/\mitobo on how to choose the names. However, it is strongly
recommended to adhere to the \alida/\mitobo naming convention as this helps to avoid
name conflicts. In \mitobo all variables start with prefix 'MITOBO'. 
Likewise in \alida all variables start with prefix 'ALIDA'. Note that some of the \alida specific
variables are also of interest in \mitobo.
The second part of the name is usually the operator class using the variable, and the third part is the actual
variable name.

\begin{center}
\fbox{\parbox{0.95\textwidth}{
\underline{Example:}\\
Imagine an operator called 'DummyOperator' which defines a variable 'Input'.\\
The environment variable that will be checked
by the operator is then

\centerline{\texttt{MITOBO\_DUMMYOPERATOR\_INPUT}}

\vspace*{0.25cm}
Obeying the naming conventions for ImageJ properties the\\
corresponding preference and also the JVM property is named\\[-0.2cm]

\centerline{\texttt{mitobo.dummyoperator.input}}
}}
\end{center}

Besides operator specific variables there may exist variables of global interest shared by different operators. In their
names the second part is simply missing, like in\\\\
\centerline{\texttt{MITOBO\_IMAGEDIR}\hspace*{1cm} or \hspace*{1cm}{\tt mitobo.imagedir},}\\\\
repectively. When defining such variables, however, special care has to be taken for
ensuring that such variables are interpreted the
same wherever they are used. And even more important, it needs to be thoroughly
verified that the variables were not
already defined elsewhere which might result in strange behavior of certain operators.

\section{List of Important Variables and Properties}
\label{sec:importantVars}
Below you find a list of variables and properties of presumably common interest.
\begin{itemize}
\item ALIDA\_OPRUNNER\_FAVORITEOPS
	\begin{itemize}
	    \item used by: {\tt ALDOpRunnerGUI}, {\tt ALDGrappaRunner}, {\tt Op\_Runner}, {\tt Grappa\_Editor}
	    \item description: configures which operators should automatically be unfolded in the operator selection menus of the 
	    	graphical operator runners and Grappa upon start-up; it should be set to a filename, and the corresponding file should 
	    	contain one operator per line, e.g.\\
	    	{\tt 
				de.unihalle.informatik.Alida.demo.ALDCalcMeanVector\\
				de.unihalle.informatik.MiToBo.tools.image.ImageDimensionReducer\\
				\ldots
				}
	\end{itemize}
\item ALIDA\_OPRUNNER\_OPERATORPATH
        \begin{itemize}
            \item used by: {\tt ALDOpRunnerGUI}, {\tt ALDGrappaRunner}, {\tt Op\_Runner}, {\tt Grappa\_Editor}
		\item	description: specifies colon separated list of packages; 
        Each package and all its sub-packages 
        is searched for operators in the classpath.
        These operators are incorporated in the tree of available operators
        in the graphical user interface.
        This feature is useful to incorporate operators which are not compiled
        but just added as within a jar-archive.
	\end{itemize}

\item ALIDA\_OPRUNNER\_LEVEL
	\begin{itemize}
	    \item used by: {\tt ALDOpRunnerGUI}, {\tt ALDGrappaRunner}, {\tt Op\_Runner}, {\tt Grappa\_Editor}
	    \item description: configures which set of operators is to be displayed initially in the selection menu; 
	    	possible options are either all available operators ('standard') or just the ones categorized as being easier 
	    	to use ('application') 
	\end{itemize}
\item ALIDA\_OPRUNNER\_WORKFLOWPATH
	\begin{itemize}
	    \item used by: all graphical and commandline operator runners, and by Grappa
	    \item description: specifies a directory where the runners are searching for additional workflows that are to be registered
	    	by the framework
	\end{itemize}
\item ALIDA\_VERSIONPROVIDER\_CLASS
	\begin{itemize}
	    \item used by: Framework
	    \item description: class used for acquiring software versions for process documentation; the class must extend the base class 
	    	{\tt ALDVersionProvider} to be found in the \alida package {\tt de.unihalle.informatik.Alida.version}  
	\end{itemize}
\item MITOBO\_IMAGEDIR
	\begin{itemize}
	    \item used by: {\tt Open\_Image\_MTB} , {\tt Save\_Image\_MTB}
	    \item description: Directory where images are expected; checked if the two variables\\
	    	MITOBO\_OPENDIR and/or MITOBO\_SAVEDIR	are not set
	\end{itemize}
\item MITOBO\_OPENDIR
	\begin{itemize}
	    \item used by: {\tt Open\_Image\_MTB}
	    \item description: directory where image browsing starts the first time 
	\end{itemize}
\item MITOBO\_SAVEDIR
	\begin{itemize}
	    \item used by: {\tt Save\_Image\_MTB}
	    \item description: directory where file selection starts the first time 
	\end{itemize}
\end{itemize}
