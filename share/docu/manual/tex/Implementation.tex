\section{Using Operators in Your Code}
\label{sec:callOp}
\alida defines a unique invocation mechanism for operators which needs to be applied when using
operators on the programmatic level. Basically the following steps are necessary to use a \mitobo
operator in your own code:
\begin{enumerate}
  \item instantiate an object of the desired operator class
  \item set the operator parameters
  \item execute the operator by calling its {\tt 'runOp(\ldots)'} method
  \item get the results
\end{enumerate}

\begin{figure*}[tbp]
\centering
\lstinputlisting[linerange={133-142},basicstyle=\small, xrightmargin=.1\textwidth,
xleftmargin=.1\textwidth]{../code/ImgOpen.java }
\caption{\label{exa:openExample}Example of a hypothetical function applying an opening to an 
image which is implemented based on \mitobo operators for morphological operations.}
\end{figure*}
In Fig.~\ref{exa:openExample} an examplary function calling \mitobo operators is shown based on 
which we will now outline the different steps listed above in detail. Suppose that the function 
that is to be realized based on \mitobo operators should perform an opening operation on an image,
i.e.~first do an erosion, then a dilation. \mitobo includes two operators for these tasks, the
erosion operator {\tt 'ImgErode'} and the dilation operator {\tt 'ImgDilate'}. Given these 
operators our opening function can be implemented as follows. First of all we instantiate an 
object of class {\tt 'ImgErode'} (line $2$ in Fig.~\ref{exa:openExample}) and specify the 
parameters using its {\tt 'setParameter(\ldots)'} method (lines $3-4$). 
The method actually takes as input the name of
the parameter to be set and the corresponding value. In line $5$ we are running the operator 
calling its {\tt 'runOp(\ldots)'} method. Note that this method is the {\em only} way of 
executing an operator. Although there are different versions of the method dedicated
to different strategies on how to document the call to the operator in the processing history
all of them basically do some logging for the processing history and then call the operator's
{\tt 'operate()'} method. Please refer to the \alida guide for further details. 

After eroding the input image we want to apply a
dilation to the result image of the erosion. To this end we instantiate an object of type
{\tt 'ImgDilate'} (line $6$). This operator class, for convenience, offers a constructor which 
already takes the parameter values as arguments. Consequently, the call of the constructor is 
sufficient in this case to readily configure the operator. In line $7$ the dilation operator is 
finally executed, and subsequently the result image is fetched from the operator and returned by
the function that was just implemented.

\section{Implementing Operators}
\label{sec:makeOp}
When the \mitobo operator runners or Grappa are required to invoke an operator, they basically 
follow the steps outlined in the previous section that have to be performed for using an operator on 
the programmatic level. First they instantiate an object of the corresponding class, then they set 
the operator's parameters to the values queried graphically or via commandline from the user,
and finally they call the {\tt 'runOp(\ldots)'} method for executing the operator. To enable this
procedure generically for all operators the implementation of an operator requires the following
things to be done:
\begin{enumerate}
  \item implementation of a public default constructor for the new operator class without any 
  	parameters as this is the only constructor of a class which can be called generically
  \item definition of the operator parameters, in \alida/\mitobo this is accomplished
  	by annotating corresponding member variables of the operator class
  \item implementation of the method {\tt 'operate()'} which should contain the functionality
  \item annotation of the operator class as a whole
\end{enumerate}
The last step is necessary to enable \alida to automatically register operators upon
start-up, e.g., to make them available to the graphical operator runner or as nodes in Grappa.

In Fig.~\ref{exa:opExample} a code snippet of a prototypical operator class is shown. The code 
is a slightly simplified version of the operator class 
{\tt 'de.unihalle.informa\-tik.MiToBo.morpholo\-gy.ImgErode'} shipped with \mitobo.

An operator in \mitobo is basically the implementation of a class extending the base class 
{\tt 'de.unihalle.informatik.MiToBo.core.operator.MTBOperator'} common for all operators (which 
by itself extends {\tt 'de.unihalle.informatik.Alida.operator.ALDOperator'}). Accordingly
from line $4$ of the code it can be seen that the new class extends \mitobo's operator base 
class. Subsequently in lines $6-16$ the parameters of the operator are declared. They are given by 
member variables annotated as {\tt 'Parameter'}. This annotation is an \alida annotation, but 
is quite similar to the corresponding annotation used in ImageJ $2.0$. The annotation requires 
for each parameter to specify a label that, e.g., is used for automatic GUI generation. In addition,
the direction of the parameter needs to be specified, i.e.~if it is an input or an output parameter,
and a short description should be given. This description is for example used to generate tooltips 
for the parameters in the operator configuration windows. 

In lines $18-24$ the default constructor of the class is implemented, while its main function is
to be found in lines $33-40$. The {\tt 'operate()'} function does the actual work. In this example 
it simply calls an internal method of the operator class (not shown for clarity), and stores the 
result of this function in the operator's output parameter 'resultImg'.

As mentioned above the operator class by itself needs to be annotated to be automatically 
registered by the \alida/\mitobo framework. This annotation can be seen in lines $2-3$. The 
annotation {\tt 'ALDAOperator'} has certain parameters, e.g.~a generic execution mode can be 
specified which allows to exclude operators from generic graphical or commandline execution. 
Also the category of the operator can be specified, i.e.~if it is easily applicable and of common 
interest, or if it is rather specialized and most probably not of common interest for most users
(cf.~Sec.~\ref{sec:opRunImageJ}).   
 \begin{figure*}[htbp]
\centering
\lstinputlisting[linerange={67-108},basicstyle=\small, xrightmargin=.1\textwidth,
xleftmargin=.1\textwidth]{../code/ImgErode.java }
\caption[Implementation example for a standard
ImageJ plugin]{\label{exa:opExample}Example implementation of an operator in \mitobo.}
\end{figure*}

On implementing a new operator you are basically free on how to organize your class. One thing 
you should keep in mind, however, is that the {\tt 'operate()'} method is the only method ever 
called on operator objects from outside (except from public getter and setter methods, if provided). 
Hence, all initialization required by the 
operator should be done within or at least be invoked from within this function, and never 
within any constructor. Finally, for improving usability of operators it is advisable,
even if not mandatory, to provide getter and setter methods for its parameters as these 
significantly simplify usage of the operator on the programmatic level.
