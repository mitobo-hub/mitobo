\alida and \mitobo provide certain classes not directly related to image processing, however, useful for doing things like time
measurements or operator configuration. Such tools can usually be found in the packages {\tt de.unihalle.informatik.Alida.helpers}
and {\tt de.unihalle.informatik.MiToBo.core.helpers}, respectively.

\section{Operator Configuration}
\label{tools:plugconf}
For user specific configuration of operators \alida supports environment variables and JVM properties, while \mitobo as well supports 
ImageJ preferences (see also Chap. \ref{chap:config}). For accessing environment variables and properties/preferences \alida 
provides the class \texttt{de.unihalle.informatik.Alida.helpers.ALDEnvironmentConfig} which supports easy access to these variables
and properties. It basically defines the following methods:
\begin{itemize}
    \item {\small \texttt{public static void getConfigValue( String operator,	String propname )}}\\[0.2cm]
    	reads the value of the environment variable or property with name \mbox{'ALIDA\_operator\_propname'} or 'alida.operator.propname',
    	respectively; it follows the formerly defined 
    	priority ordering of the different configuration options, i.e.~first looks for an environment variable with the given name, 
    	then checks for JVM properties 
\end{itemize} 
If not all options should be checked, the following methods can be used alternatively:
\begin{itemize}
    \item {\small \texttt{public static String getEnvVarValue( String operator, String envVariable )}}\\[0.2cm] 
    	This method allows to directly read (only) environment variables. 
    \item {\small \texttt{public static String getJVMPropValue( String operator, String envVariable )}}\\[0.2cm] 
    	This method allows to directly read (only) JVM properties. 
\end{itemize}
\mitobo includes the class \texttt{de.unihalle.informatik.MiToBo.core.helpers.MTBEnviron\-mentConfig} extending its \alida superclass
by additional methods for accessing ImageJ properties:
\begin{itemize}
    \item {\small \texttt{public static String getImageJPropValue( String operator, String envVariable )}}\\[0.2cm] 
    	This method allows to directly read ImageJ preferences.  
    \item {\small \texttt{public static void setImageJPref( String operator, String envVar, String val )}}\\[0.2cm]
    	This method allows to set a preference in the ImageJ configuration file.
    	It is saved to the user specific ImageJ configuration file (in ImageJ usually $\;\tilde{ }$/.imagej/IJ\_Prefs.txt, but this
    	may be different in ImageJ $2.0$). Note that for actually saving the settings it is required to have the ImageJ GUI open,
    	because only on closing the GUI the preferences are actually written to the configuration file.
\end{itemize}
Note that in all cases the prefix 'mitobo.' for properties and preferences or 'MITOBO\_' for environment variables,
respectively, is internally added to the variable and property names. Likewise \mitobo overwrites the superclass methods of \alida 
as introduced above to automatically
set the \mitobo prefixes in the variable and property names. Anyway, the programmer usually does not need to pay
attention on this feature as long as he or she follows the standard naming conventions in \alida and \mitobo.